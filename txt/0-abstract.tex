\documentclass[../main.tex]{subfiles}

\begin{document}

\chapter*{\myAbstractTitle}

This thesis presents a case study on the implementation of a centralized logging solution with Graylog on Kubernetes within a mid-size company of 500 employees. The study explores the practical maintainability of this solution over time and assesses its perceived value in the real-world use case. Many organizations struggle with logging solutions that are either too complex to maintain or lack necessary features. This research aims to bridge that gap by evaluating Graylog’s feasibility in a production environment.

The maintainability of the system is analyzed through simulated updates over a two-year period (2023–2024). To assess user perception, the Kano model approach is employed, identifying key features through interviews with employees involved in the selected use cases. The results indicate that maintenance costs remain low, provided the responsible personnel have prior knowledge of the relevant technologies. Furthermore, the questionnaire highlights which features are most valued by potential users, offering insights into optimizing tool selection.

Overall, the findings demonstrate that Graylog is a viable centralized logging solution, balancing ease of maintenance and user satisfaction. These insights can serve as a guideline for other organizations considering similar implementations.

\end{document}
