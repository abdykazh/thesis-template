\documentclass[../main.tex]{subfiles}

\begin{document}

\chapter{Introduction}

\gls{fisp} \cite{fisp} is a German IT service provider that offers outsourcing, consulting, development, and integration services for business applications in the financial sector. \gls{fisp} is a wholly owned subsidiary of \gls{fi}, the digitalization partner of the Sparkassen-Finanzgruppe \cite{spfi}, with a team of over 500.  The key clients include state banks, savings banks and affiliated partners within the Sparkassen-Finanzgruppe across Germany. Established in 1997, the company has undergone multiple strategic restructurings, leading to a technologically diverse infrastructure that integrates legacy systems with modern advancements.

The company consists of dozens of teams across three locations, each responsible for developing or managing client applications. These teams operate independently, using different technologies, deployment strategies, and serving various clients.

For internal and external projects, the infrastructure team provides technical services and advisory expertise. This team manages Kubernetes clusters on-premises, offering the necessary infrastructure while individual teams handle the deployment and ongoing operation of their applications on cluster. The Kubernetes environment is deployed using Rancher, an open-source Kubernetes management platform. At the moment, only a handful of projects run on the cluster, with the expectation that more applications will be containerized and run on Kubernetes in the future.

\section{Problem Statement}

Logs are generated by various Kubernetes components, including the API server, Kubelet, container runtime environment, and applications running on the cluster. These logs contain critical data such as application performance, errors, warnings, and system events, which developers, application owners, and infrastructure teams rely on for monitoring, troubleshooting, and maintaining application health.

However, the company lacked a centralized logging solution. Employees had to manually navigate through different locations to find relevant applications and view logs individually. A centralized logging system simplified this process by collecting, processing, and making log data accessible from a single location. This eliminated the need to search across multiple sources and provided additional benefits such as filtering, visualization, analysis, and operational insights.

The goal of this thesis was to propose and evaluate a solution to the company's logging challenges. As specified by the company, the solution was based on Graylog \cite{graylog} and the Logging Operator \cite{logoperator}.

\section{Research Questions}

\begin{itemize}

\item[Q1.] What are the benefits and costs associated with the proposed logging solution and under what conditions is this centralized logging solution (Graylog \& Logging Operator) preferable?

\item[Q2.] How do employees perceive the usability, value, and benefits of Graylog's capabilities in their workflow?
%what do you mean by this question? Q3 including its subpoints need to be edited
\item[Q3.] What insights can be gained from analyzing the company's application log queries?

\item Which measurement methods are best suited to ensure that the company's actual goals are met?
Are there second-order effects with "really good" and "not-so-great" systems?

\item Is the approach of achieving "complete log understanding" a feasible strategy in complex systems?
Can a feedback loop to application developers regarding log quality be established?

\end{itemize}

\section{Research Objectives}

\begin{itemize}
\item[1.] Deploy a centralized logging solution in the company's environment using Logging Operator, Graylog, and OpenSearch on Kubernetes (Rancher).
\item[2.] Assess the maintainability of the implemented solution.
\item[3.] Evaluate the perceived usability and value for employees. 
\item[4.] Examine which of Graylog's capabilities benefit employees.
\end{itemize}

\end{document}
